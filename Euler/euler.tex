\documentclass[fleqn]{article}

\usepackage{amsmath}
\usepackage[top=2mm, bottom=2mm, left=2mm, right=30mm]{geometry}
\usepackage{forloop}
\begin{document}

Problem 44.  

Pentagon number.  Given a number, is it a pentagon ?

\begin{flalign}
P_n = \frac{n(3n-1)}{2}, \\
2*P_n = n(3n-1), \\
2*P_n = 3n^2-n, \\
0 = 3n^2-n-2*P_n, \\
\frac{1+\sqrt{1^2 - 4 * 3 * -2*P_n }}{2 * 3}, \\
\frac{1+\sqrt{1^2 + 24P_n }}{6}
\end{flalign}

Problem 50.


\newcounter{index}
\begin{flalign}
\begin{tabular}{*{20}{c|}}
 & 0 & 1 & 2 & 3 & 4 & 5 & 6 & 7 & 8 & 9 & 10 & 11 \\
\hline
Max  & 0 & 0 & 1 & 1 & 0 & 1 & 0 & 1 & 0 & 0 & 0 & 1  \\
Prev & 0 & 0 & 0 & 0 & 0 & 0 & 0 & 0 & 0 & 0 & 0 & 0  \\
\end{tabular}
\end{flalign}
Then loop through each prime starting at lowest.  2
Only update if it is a new maximum.  Keep old array
\begin{flalign}
\begin{tabular}{*{20}{c|}}
 & 0 & 1 & 2 & 3 & 4 & 5 & 6 & 7 & 8 & 9 & 10 & 11 \\
\hline
Max  & 0 & 0 & 1 & 0 & 0 & 0 & 0 & 0 & 0 & 0 & 0 & 0  \\
Prev & 0 & -1& 0& -1& -1& -1& -1& -1& -1& -1& -1& -1  \\
\end{tabular}
\end{flalign}

3
\begin{flalign}
\begin{tabular}{*{20}{c|}}
 & 0 & 1 & 2 & 3 & 4 & 5 & 6 & 7 & 8 & 9 & 10 & 11 \\
\hline
Max  & 0 & 0 & 1 & 1 & 0 & 2 & 0 & 0 & 0 & 0 & 0 & 0  \\
Prev & 0 & -1& 0& 0& -1& 2& -1& -1& -1& -1& -1& -1  \\
\end{tabular}
\end{flalign}

5
\begin{flalign}
\begin{tabular}{*{20}{c|}}
 & 0 & 1 & 2 & 3 & 4 & 5 & 6 & 7 & 8 & 9 & 10 & 11 \\
\hline
Max  & 0 & 0 & 1 & 1 & 0 & 2 & 0 & 2 & 2 & 0 & 3 & 0  \\
Prev & 0 & -1& 0& 0& -1& 2& -1& 2& 3 & -1& 5& -1  \\
\end{tabular}
\end{flalign}

\end{document}
